\setcounter{page}{2}
\addchap{Введение}
Время работы - одна из основных характеристик, влияющих на оценку алгоритма. Существует множество способов улучшения этой характеристики. В данной работе будет рассмотрен один из них -- организация вычислительного конвейера. Такой подход предполагает разделение алгоритма на несколько независимых этапов, выходные данные каждого из которых являются входными этапами следующего. 

В представленной работе исследуется реализация вычислительного конвейера на параллельных процессах.
В качестве алгоритма, который будет декомпозирован на этапы и обернут в вычислительный конвейер, был выбран алгоритм Бойера -- Мура -- Хорспула для поиска подстрок. 

Цель лабораторной работы -- исследование параллельных конвейерных вычислений. Для достижения поставленной цели необходимо выполнить следующие задачи:
\begin{itemize}
	\item исследовать организацию конвейерной обработки данных;
	\item разработать ПО, реализующее конвейер с количеством лент не менее трех в однопоточной и многопоточной среде;
	\item исследовать зависимость времени работы конвейера от количества потоков, на которых он работает. 
\end{itemize}
Результаты сравнительного анализа будут приведены в виде таблиц, из которых можно будет сделать вывод об эффективности работы алгоритма Бойера -- Мура на многопоточном конвейере, предложенной в данной работе.