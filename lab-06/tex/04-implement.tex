\chapter{Технологический раздел}\label{sec:impl}

Раздел содержит листинги реализованных алгоритмов, требование к ПО и тестирование реализованного программного обеспечения. Также раздел включает в себя описание системы автоматической параметризации.

\section{Требования к ПО}

Программное обеспечение должно удовлетворять следующим требованиям:
\begin{itemize}
	\item программа принимает на вход количество дней, симметричную матрицу смежности графа и регулируемые параметры;
	\item программа выдает массив кратчайших путей, выходящих из каждой вершины графа.
\end{itemize}

\section{Средства реализации} 
Для реализации ПО был выбран компилируемый многопоточный язык программирования Golang, поскольку язык отличается легкой и быстрой сборкой программ и автоматическим управлением памяти.
В качестве среды разработки была выбрана среда VS Code, написание сценариев осуществлялось утилитой make.
Скрипты для автоматизации были написаны на языке python.

\section{Листинги кода}

\subsection{Муравьиный алгоритм}

Листинг \ref{lst:types} демонстрирует типы данных для алгоритма муравьиной колонии.

\captionsetup{singlelinecheck = false, justification=raggedright}
\begin{lstlisting}[label=lst:types,caption=Структуры данных]
type Ant struct {
	colony  *Colony
	tour     [][]int
	memory   [][]bool
	position int
	start    int
}
type Colony struct {
	gr_matrix    [][]int
	phero_matrix [][]float64
	iters        int
	conf         Config
}
type Config struct {
	TRACE_WEIGHT float64 // alpha
	TOUR_VISIB   float64 // beta
	EVAP_RATE    float64 // p
	Q        	 float64 // q
	PHERO_INIT   float64 // start pheromone conc
	PHERO_EPS    float64
}
\end{lstlisting}

Листинг \ref{lst:colony-init} демонстрирует аллокацию структуры муравьиной колонии.
\begin{lstlisting}[label=lst:colony-init,caption=Аллокация колонии]
func makeColony(gr_matrix [][]int, iters int, 
				conf_filename string) *Colony {
	colony := new(Colony)
	conf := ParceConfigFile(conf_filename)
	if (conf == nil) {
		return nil
	} 
	colony.conf = *conf

	colony.gr_matrix = gr_matrix
	colony.iters = iters
	colony.phero_matrix = make([][]float64, len(colony.gr_matrix))
	
	for i := 0; i < len(colony.gr_matrix); i++ {
		colony.phero_matrix[i] = make([]float64, len(colony.gr_matrix[i]))
		for j := range colony.phero_matrix[i] {
			colony.phero_matrix[i][j] = colony.conf.PHERO_INIT
		}
	}
	return colony
}
\end{lstlisting}

Листинг \ref{lst:place-ant} демонстрирует размещение муравья на точке в графе.
\begin{lstlisting}[label=lst:place-ant,caption=Размещение мурвья]
func (col *Colony) placeAnt(start int) *Ant {
	ant := new(Ant)
	ant.colony   = col
	ant.tour     = make([][]int, len(col.gr_matrix))
	ant.memory   = make([][]bool, len(col.gr_matrix))
	ant.position = start
	ant.start    = start
	
	for i := range col.gr_matrix {
		ant.tour[i] = make([]int, len(col.gr_matrix[i]))
		copy(ant.tour[i], col.gr_matrix[i])
	}

	for i := range ant.memory {
		ant.memory[i] = make([]bool, len(col.gr_matrix))
	}
	return ant
}
\end{lstlisting}

Пользователь взаимодействует с функцией, продемонстрированной на листинге \ref{lst:wrap}.
\begin{lstlisting}[label=lst:wrap,caption=Алгоритм муравьиной колонии]
func ACOWrapper(gr_matrix [][]int, iters int, 
				conf_filename string) *[]int {
	shortest := make([]int, len(gr_matrix))
	colony := makeColony(gr_matrix, iters, conf_filename)
	if (colony == nil) {
		return nil
	}
	
	for i := 0; i < colony.iters; i++ {
		for j := 0; j < len(colony.gr_matrix); j++ {
			ant := colony.placeAnt(j)
			ant.startTour()
			current := ant.tourLength()
			
			if current < shortest[j] || shortest[j] == 0 {
				shortest[j] = current
			}
		}
	}
	return &shortest
}
\end{lstlisting}
Начало перемещения муравья по графу происходит согласно функции \ref{lst:ant-start}:

\begin{lstlisting}[label=lst:ant-start,caption=Перемещение муравья по графу]
func (ant *Ant) startTour() {
	next := -1
	
	for flag := true; flag; flag = (next != -1) {
		vis := ant.getVision()
		
		next = fortuneWheel(vis)
		
		if (next != -1) {
			ant.tourAnt(next)
			ant.updatePhTrace()	
		}
	} 
}
\end{lstlisting}

Вероятностный массив, полученный согласно правилу \ref{ant-prob}, вычисляется в следующей функции(\ref{lst:prob})
\begin{lstlisting}[label=lst:prob,caption=Получение массива зрения для муравья]
func (ant *Ant) getVision() []float64 {
	var (
	vis = make([]float64, 0)
	denom = 0.0
	)
	
	for i, l := range ant.tour[ant.position] {
		if l != 0 {
			term := math.Pow((1.0 / float64(l)), 
			ant.colony.conf.TRACE_WEIGHT) * math.Pow
			(ant.colony.phero_matrix[ant.position][i], 
			ant.colony.conf.TOUR_VISIB)
			
			denom += term
			vis = append(vis, term)
		} else {
			vis = append(vis, 0)
		}
	}
	
	for i := 0; i < len(vis); i++ {
		vis[i] /= denom
	}
	return vis
}	
\end{lstlisting}
Листинг \ref{lst:ant-wheel} демонстрирует метод выбора следующего города.
\begin{lstlisting}[label=lst:ant-wheel,caption=Выбор следующего города]
func fortuneWheel(vis []float64) int {
	var (
	cumul_sum = make([]float64, len(vis))
	coin = rand.New(rand.NewSource(time.Now().UnixNano())).Float64()
	chosen = -1
	)
	
	for i := 0; i < len(vis); i++ {
		for j := i; j < len(vis); j++ {
			cumul_sum[i] += vis[j]
		}
	}

	for i := 0; i < len(cumul_sum); i++ {
		if i == len(cumul_sum) - 1 {
			if 0.0 <= coin && coin <= cumul_sum[i] {
				chosen = i
			}
		} else {
			if cumul_sum[i + 1] < coin && coin <= cumul_sum[i] {
				chosen = i
			}
		}
	}

	return chosen
}	
\end{lstlisting}
Поведение муравья после выбора города описано на листинге \ref{lst:ant-tour}
\begin{lstlisting}[label=lst:ant-tour,caption=Путь муравья в следующий город]
func (ant *Ant) tourAnt(root int) {
	for i := range ant.tour[ant.position] {
		ant.tour[i][ant.position] = 0
		ant.tour[ant.position][i] = 0
	}

	ant.memory[ant.position][root] = true	
	ant.position = root
}
\end{lstlisting}
Листинг \ref{lst:phero} демонстрирует обновление феромона согласно формуле \ref{pheromone}.\newpage
\begin{lstlisting}[label=lst:phero,caption=Обновление феромона]
func (ant *Ant) updatePhTrace() {
	delta := 0.0
	
	for i := 0; i < len(ant.colony.phero_matrix); i++ {
		for j, phero := range ant.colony.phero_matrix[i] {
			if ant.colony.gr_matrix[i][j] != 0 {
				if ant.memory[i][j] {
					delta = ant.colony.conf.Q / 
						float64(ant.colony.gr_matrix[i][j])
				} else {
					delta = 0
				}
				ant.colony.phero_matrix[i][j] = 
				(1 - ant.colony.conf.EVAP_RATE) * (float64(phero) + delta)
				ant.colony.phero_matrix[j][i] = 
				(1 - ant.colony.conf.EVAP_RATE) * (float64(phero) + delta)
			}
			if ant.colony.phero_matrix[i][j] <= 0 {
				ant.colony.phero_matrix[i][j] = ant.colony.conf.PHERO_EPS
				ant.colony.phero_matrix[j][i] = ant.colony.conf.PHERO_EPS
			}
		}
	}
}
\end{lstlisting}
После прохождения маршрута муравей вычисляет длину пройденного пути следующим образом (\ref{lst:len}):
\begin{lstlisting}[label=lst:len,caption=Длина маршрута]
func (ant *Ant) tourLength() int {
	var (
	length = 0
	last   = ant.position
	)
	
	for i, r := range ant.memory {
		for root, passed := range r {
			if passed {
				length += ant.colony.gr_matrix[i][root]
			}
		}
	}
	length += ant.colony.gr_matrix[ant.start][last]
	
	return length
}
\end{lstlisting}
\subsection{Алгоритм полного перебора}
Листинг \ref{lst:brute} демонстрирует алгоритм полного перебора.
\begin{lstlisting}[label=lst:brute,caption=]
func BruteForce(gr_matrix [][]int) []int {
	var (
	roots = make([]int, 0)
	graph_len = len(gr_matrix)
	min_path = -1
	)
	for source := 0; source < graph_len; source++ {
		cur_roots := make([]int, 0)
		for i := 0; i < graph_len; i++ {
			if i != source {
				cur_roots = append(cur_roots, i)
			}
		}
		next_permutation := permutations(cur_roots)
		for _, perm := range next_permutation {
			curr_cost := 0
			k := source
			
			for _, j := range perm {
				curr_cost += gr_matrix[k][j]
				k = j
			}
			curr_cost += gr_matrix[k][source]
			if curr_cost < min_path || min_path == -1 {
				min_path = curr_cost
			}
		}
		roots = append(roots, min_path)
		min_path = -1
	}
	return roots
}	
\end{lstlisting}
Получение всех перестановок происходит согласно функциям, демонстрируемым на листингах \ref{lst:prem} и \ref{lst:prem-helper}.
\begin{lstlisting}[label=lst:prem,caption=Получение всех перестановок]
func permutations(path []int) [][]int {
	var res [][]int
	__permutationHelper(path, &res, 0)
	return res
}
\end{lstlisting}
\begin{lstlisting}[label=lst:prem-helper,caption=Получение одной перестановки]
func __permutationHelper(path []int, res *[][]int, k int) {
	for i := k; i < len(path); i++ {
		path[i], path[k] = path[k], path[i]
		__permutationHelper(path, res, k + 1)
		path[k], path[i] = path[i], path[k]
	}
	if (k == len(path) - 1) {
		r := make([]int, len(path))
		copy(r, path)
		*res = append(*res, r)
		return
	}
}	
\end{lstlisting}

\subsection{Автоматическая параметризация}

скрипт автоматической перестановки генерирует все комбинации параметров в заданном диапазоне и помещает в конфигурационный файл. Функция генерации параметров имеет следующий вид (\ref{lst:script}):
\begin{lstlisting}[label=lst:script,caption=Генерация параметров]
def form_coefs():
	coefs_gen = []
	for i in range(11):
	stack = []
	alpha = round(0.0 + 0.1 * i, 2)
	betta = round(1 - alpha, 2)
	
	stack.append([round(0.0 + 0.1 * j, 2) for j in range(11)])
	stack.append([40])                                         # q
	stack.append([0.1])                                        # init
	stack.append([0.01])                                       # eps
	cprod = list(itertools.product(*stack))
	coefs_cur = []
	for x in cprod:
		x = list(x)
		x.insert(0, alpha)
		x.insert(1, betta)
		coefs_cur.append(x)
		coefs_gen.append(coefs_cur)
	
	return coefs_gen
\end{lstlisting}

\section{Тестирование ПО}
Результаты тестирования ПО приведены в таблице \ref{table:testing}. 
\begin{table}[H]
	\centering
	\captionsetup{singlelinecheck = false, justification=raggedleft}
	\renewcommand{\arraystretch}{2}
	\caption{Тестирование ПО}
	
	\begin{tabular}{||c|c|c||}
		\hline
		\textit{Матрица} & \textit{Полный перебор} & \textit{Муравьиный алгоритм} \\ \hline\hline
		\begin{tabular}[c]{@{}c@{}}$ \begin{matrix}[.5]\\     0 & 7 & 8 & 7 & 9  \\\\     7 & 0 & 8 & 3 & 6  \\\\     8 & 8 & 0 & 8 & 6  \\\\     7 & 3 & 8 & 0 & 8  \\\\     9 & 6 & 6 & 8 & 0  \\\\ \end{matrix}  $\end{tabular} & 30 30 30 30 30 & 30 30 30 30 30 \\ \arrayrulecolor{gray35}\hline
		\begin{tabular}[c]{@{}c@{}}$ \begin{matrix}[.5]\\ 0 & 3 & 1 & 6 & 8 \\\\ 3 & 0 & 4 & 1 & 2 \\\\ 1 & 4 & 0 & 5 & 2 \\\\ 6 & 1 & 5 & 0 & 6 \\\\ 8 & 2 & 2 & 6 & 0 \\\\ \end{matrix}  $\end{tabular} & 12 12 12 12 12 & 12 12 12 12 12 \\\arrayrulecolor{gray35}\hline
		\begin{tabular}[c]{@{}c@{}}$ \begin{matrix}[.5]\\ 0  & 10  & 15  & 20  \\\\ 10 &  0  & 35  & 25  \\\\ 15 &  35 &  0  & 30  \\\\ 20 &  25 &  30 &  0  \\\\  \\ \end{matrix}  $\end{tabular} & 80 80 80 80 & 80 80 80 80 \\ \hline
	\end{tabular}
	\label{table:testing}
\end{table}

\section{Вывод}
Было написано и протестировано программное обеспечение для решения поставленной задачи.