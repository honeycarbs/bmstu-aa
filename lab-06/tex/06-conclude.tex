\addchap{Заключение}
В ходе лабораторной работы были реализованы алгоритмы полного перебора и муравьиной колонии, были разработаны два класса эквивалентности для параметризации муравьиного алгоритма и была проведена параметризация. Был проведен сравнительный анализ алгоритмов. 


Муравьиные алгоритмы обеспечивают качественное, насколько это возможно, решение комбинаторных задач, для которых не существует быстрых полиномиальных решений. На исследуемых классах данных алгоритм показал оптимизацию в 400 раз. Проведенное исследование позволяет рекомендовать применение муравьиного алгоритма вместо алгоритма полного перебора для решения задачи коммивояжера, хотя алгоритм полного перебора является более универсальным -- для него не нужно проводить параметризацию, он дает гарантированно точный результат, хоть и за большее время. 
Опираясь на проведенное исследование, можно сделать вывод, что если критерием оценки задачи коммивояжера является универсальность, то следует использовать алгоритм полного перебора. Однако, если задача должна быть решена на определенном классе данных и критерием оценки является скорость выполнения, то для решения стоит выбрать алгоритм муравьиной колонии. 