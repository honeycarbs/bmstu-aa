\chapter{Технологический раздел}

\section{Требования к ПО}

Программное обеспечение должно удовлетворять следующим требованиям:
\begin{itemize}
	\item Программа получает на вход с клавиатуры две матрицы размеров в пределах $10000 \times 10000$ либо получает два числа -- размерность матрицы в пределах $10000$;
	\item Программа выдает матрицу - произведение двух полученных матриц.
\end{itemize}

\section{Средства реализации} 
Для реализации ПО был выбран язык программирования Golang, поскольку язык отличается легкой и быстрой сборкой программ и автоматическим управлением памяти. Также язык обеспечивает возможность вставки на языке С, что позволит исследовать реализованные алгоритмы с точки зрения временных затрат.

\section{Листинги кода}
Листинг \ref{lst:algebr} демонстрирует классический алгоритм умножения. 
\captionsetup{singlelinecheck = false, justification=raggedright}
\begin{lstlisting}[label=lst:algebr,caption=Классический алгоритм умножения]
func Algebraic(mtx_left, mtx_right mtx.Matrix) mtx.Matrix {
	var product mtx.Matrix
	product = mtx.Allocate(mtx_left.Rows, mtx_right.Cols)
	
	for i := 0; i < product.Rows; i++ {
		for j := 0; j < product.Cols; j++ {
			for k := 0; k < mtx_right.Rows; k++ {
				product.Values[i][j] += mtx_left.Values[i][k] +
				 * mtx_right.Values[k][j]
			}
		}
	}
	
	return product
}
\end{lstlisting}
Листинг \ref{lst:winograd} -- умножение матриц алгоритмом Винограда.
\begin{lstlisting}[label=lst:winograd,caption=Алгоритм умнложения Виноградом]
func Winograd(mtx_left, mtx_right mtx.Matrix) mtx.Matrix {
	var (
	row_count, col_count_left, col_count_right int
	product mtx.Matrix
	rows_precomp, cols_precomp []int
	)
	row_count = mtx_left.Rows
	col_count_left = mtx_left.Cols
	col_count_right = mtx_right.Cols
	
	product = mtx.Allocate(row_count, col_count_right)
	
	rows_precomp = __precomputeRows(rows_precomp, mtx_left,
									row_count, col_count_left)
	cols_precomp = __precomputeCols(cols_precomp, mtx_right,
									col_count_left, col_count_right)
	for  i := 0; i < row_count; i++ {
		for j := 0; j < col_count_right; j++ {
			product.Values[i][j]= -(rows_precomp[i] + cols_precomp[j])
			for k := 1; k < col_count_left; k+= 2 {
				product.Values[i][j] = product.Values[i][j] +
				+ (mtx_left.Values[i][k - 1] + +
				+ mtx_right.Values[k][j]) * +
				(mtx_left.Values[i][k] + mtx_right.Values[k - 1][j])
			}
		}
	}
	if (col_count_left % 2 != 0) {
	for  i := 0; i < row_count; i++ {
		for j := 0; j < col_count_right; j++ {
			product.Values[i][j] = product.Values[i][j] + +
			mtx_left.Values[i][col_count_left - 1] * +
			mtx_right.Values[col_count_left - 1][j]
		}
	}
}
return product
}
\end{lstlisting}	
Листинг \ref{lst:winograd-optimized} -- умножение оптимизированным согласно \ref{sect:optimize} алгоритмом винограда.
\begin{lstlisting}[label=lst:winograd-optimized,caption=Оптимизированный алгоритм умнложения Виноградом]
func WinogradOptimized(mtx_left, mtx_right mtx.Matrix) mtx.Matrix {
	var (
	row_count, col_count_left, col_count_right int
	product mtx.Matrix
	rows_precomp, cols_precomp []int
	)
	row_count = mtx_left.Rows
	col_count_left = mtx_left.Cols
	col_count_right = mtx_right.Cols
	
	product = mtx.Allocate(row_count, col_count_right)
	
	rows_precomp = 
				__precomputeRowsOptimized(rows_precomp, mtx_left,
										  row_count, col_count_left)
	cols_precomp =
				__precomputeColsOptimized(cols_precomp, mtx_right,
									      col_count_left, col_count_right)
	
	var i, j, k int
	is_odd := col_count_left % 2
	for  i = 0; i < row_count; i++ {
		for j = 0; j < col_count_right; j++ {
			product.Values[i][j]= -(rows_precomp[i] + cols_precomp[j])
			for k = 1; k < col_count_left; k+= 2 {
				product.Values[i][j] += (mtx_left.Values[i][k - 1] +
				+ mtx_right.Values[k][j]) * (mtx_left.Values[i][k] + 
				+ mtx_right.Values[k - 1][j])
			}
			if (is_odd == 1) {
				product.Values[i][j] += 
				mtx_left.Values[i][col_count_left - 1] * +
				+ mtx_right.Values[col_count_left - 1][j]
			}
		}
	}
	
	return product
}
\end{lstlisting}
Функции оптимизации, реализующие формулу \ref{math:w-inner-product} представлены на листинге \ref{lst:winograd-precomp}
\begin{lstlisting}[label=lst:winograd-precomp,caption=Функции препроцессирования для алгоримов Винограда]
func __precomputeRowsOptimized(rows_precomp []int, mtx_left mtx.Matrix,
row_count, col_count int) []int {
	rows_precomp = make([]int, row_count)
	for i := 0; i < row_count; i++ {
		for j := 1; j < col_count; j+= 2 {
			rows_precomp[i] += mtx_left.Values[i][j - 1] +
			* mtx_left.Values[i][j]
		}
	}
	return rows_precomp
}

func __precomputeColsOptimized(cols_precomp []int, mtx_right mtx.Matrix,
col_count_left, col_count_right int) []int {
	cols_precomp = make([]int, col_count_right)
	for i := 0; i < col_count_right; i++ {
		for j := 1; j < col_count_left; j+=2 {
			cols_precomp[i] += (mtx_right.Values[j - 1][i] +
			* mtx_right.Values[j][i])
		}
	}
	return cols_precomp
}
\end{lstlisting}

Вспомогательные функции для работы с многомерным целочисленным массивом представлены ниже(\ref{lst:mtx-util}):

\begin{lstlisting}[label=lst:mtx-util,caption=Пакет\, реализующий функции для работы с матрицами]
type Matrix struct {
	Rows int
	Cols int
	Values [][]int
}

func Allocate(rows, cols int) Matrix {
	var dest Matrix
	
	dest.Rows = rows
	dest.Cols = cols
	dest.Values = make([][]int, dest.Rows)
	for i := range dest.Values {
		dest.Values[i] = make([]int, dest.Cols)
	}
	
	return dest
}

func Randomize(src Matrix) {
	rand.Seed(time.Now().Unix())
	for i := 0; i < src.Rows; i++ {
		for j := 0; j < src.Cols; j++ {
			src.Values[i][j] = rand.Intn(10) - rand.Intn(10)
		}
	}
}
\end{lstlisting}

\captionsetup{singlelinecheck = false, justification=raggedleft}
\section{Тестирование ПО}
\begin{table}[H]
	\renewcommand{\arraystretch}{1.8}
	\caption{Тестовые случаи}
	\resizebox{\textwidth}{!}{
		\begin{tabular}{||c|c|c|c|c|c||}
			\hline
			\multirow{2}{*}{№} & \multirow{2}{*}{Входная матрица №1} & \multirow{2}{*}{Входная матрица №2} & \multicolumn{3}{c|}{Результат} \\ \cline{4-6} 
			&  &  & AG & CW & CW OPT \\ \hline\hline
			1 & \begin{tabular}[c]{@{}l@{}}$\begin{matrix}[0.6] \\ -3 & 5 & -1 & 7 \\\\ -8 & 2 & -2 & 1 \\\\ 0 & -3 & -4 & 0 \\\\ -6 & 0 & 5 & 1 \\\\ \end{matrix}$\end{tabular} & \begin{tabular}[c]{@{}l@{}}$\begin{matrix}[0.6] \\ -3 & 5 & -1 & 7 \\\\ -8 & 2 & -2 & 1 \\\\ 0 & -3 & -4 & 0 \\\\ -6 & 0 & 5 & 1 \\\\ \end{matrix}$\end{tabular} & \begin{tabular}[c]{@{}l@{}}$\begin{matrix}[0.6] \\ -73 & -2 & 32 & -9 \\\\ 2 & -30 & 17 & -53 \\\\ 24 & 6 & 22 & -3 \\\\ 12 & -45 & -9 & -41 \\\\ \end{matrix}$\end{tabular} & \begin{tabular}[c]{@{}l@{}}$\begin{matrix}[0.6] \\ -73 & -2 & 32 & -9 \\\\ 2 & -30 & 17 & -53 \\\\ 24 & 6 & 22 & -3 \\\\ 12 & -45 & -9 & -41 \\\\ \end{matrix}$\end{tabular} & \begin{tabular}[c]{@{}l@{}}$\begin{matrix}[0.6] \\ -73 & -2 & 32 & -9 \\\\ 2 & -30 & 17 & -53 \\\\ 24 & 6 & 22 & -3 \\\\ 12 & -45 & -9 & -41 \\\\ \end{matrix}$\end{tabular} \\ 
			2 & \begin{tabular}[c]{@{}l@{}}$\begin{matrix}[0.6] \\ 0 & 0 & -2 \\\\ 2 & 3 & 2 \\\\ -4 & 0 & -1 \\\\ \end{matrix}$\end{tabular} & \begin{tabular}[c]{@{}l@{}}$\begin{matrix}[0.6] \\ 0 & 0 & -2 \\\\ 2 & 3 & 2 \\\\ -4 & 0 & -1 \\\\ \end{matrix}$\end{tabular} & \begin{tabular}[c]{@{}l@{}}$\begin{matrix}[0.6] \\ 8 & 0 & 2 \\\\ -2 & 9 & 0 \\\\ 4 & 0 & 9 \\\\ \end{matrix}$\end{tabular} & \begin{tabular}[c]{@{}l@{}}$\begin{matrix}[0.6] \\ 8 & 0 & 2 \\\\ -2 & 9 & 0 \\\\ 4 & 0 & 9 \\\\ \end{matrix}$\end{tabular} & \begin{tabular}[c]{@{}l@{}}$\begin{matrix}[0.6] \\ 8 & 0 & 2 \\\\ -2 & 9 & 0 \\\\ 4 & 0 & 9 \\\\ \end{matrix}$\end{tabular} \\ 
			3 & \begin{tabular}[c]{@{}l@{}}$\begin{matrix}[0.6] \\ 5 & 7 \\\\ 7 & 0 \\\\ 3 & 1 \\\\ \end{matrix}$\end{tabular} & \begin{tabular}[c]{@{}l@{}}$\begin{matrix}[0.6] \\ 5 & 7 & 7 & 0 \\\\ 3 & 1 & -7 & 5 \\\\ \end{matrix}$\end{tabular} & \begin{tabular}[c]{@{}l@{}}$\begin{matrix}[0.6] \\ 46 & 42 & -14 & 35 \\\\ 35 & 49 & 49 & 0 \\\\ 18 & 22 & 14 & 5 \\\\ \end{matrix}$\end{tabular} & \begin{tabular}[c]{@{}l@{}}$\begin{matrix}[0.6] \\ 46 & 42 & -14 & 35 \\\\ 35 & 49 & 49 & 0 \\\\ 18 & 22 & 14 & 5 \\\\ \end{matrix}$\end{tabular} & \begin{tabular}[c]{@{}l@{}}$\begin{matrix}[0.6] \\ 46 & 42 & -14 & 35 \\\\ 35 & 49 & 49 & 0 \\\\ 18 & 22 & 14 & 5 \\\\ \end{matrix}$\end{tabular} \\ 
			4 & \begin{tabular}[c]{@{}l@{}}$\begin{matrix}[0.6] \\ -4 & 9 \\\\ -4 & -1 \\\\ -1 & 5 \\\\ 5 & 3 \\\\ \end{matrix}$\end{tabular} & \begin{tabular}[c]{@{}l@{}}$\begin{matrix}[0.6] \\ -4 & 9 & -4 \\\\ -1 & -1 & 5 \\\\ \end{matrix}$\end{tabular} & \begin{tabular}[c]{@{}l@{}}$\begin{matrix}[0.6] \\ 7 & -45 & 61 \\\\ 17 & -35 & 11 \\\\ -1 & -14 & 29 \\\\ -23 & 42 & -5 \\\\ \end{matrix}$\end{tabular} & \begin{tabular}[c]{@{}l@{}}$\begin{matrix}[0.6] \\ 7 & -45 & 61 \\\\ 17 & -35 & 11 \\\\ -1 & -14 & 29 \\\\ -23 & 42 & -5 \\\\ \end{matrix}$\end{tabular} & \begin{tabular}[c]{@{}l@{}}$\begin{matrix}[0.6] \\ 7 & -45 & 61 \\\\ 17 & -35 & 11 \\\\ -1 & -14 & 29 \\\\ -23 & 42 & -5 \\\\ \end{matrix}$\end{tabular} \\ \hline
		\end{tabular}
	}
\label{table:testing}
\end{table}
\section{Вывод}
Было написано и протестировано программное обеспечение для решения поставленной задачи.