\setcounter{page}{2}
\addchap{Введение}
Разработка и совершенствование матричных алгоритмов является важнейшей алгоритмической задачей. Непосредственное применение классического матричного умножения требует времени порядка $O(n^3)$. Однако существуют алгоритмы умножения матриц, работающие быстрее очевидного. В линейной алгебре алгоритм Копперсмита – Винограда\cite{winograd-origin}, названный в честь Д. Копперсмита и Ш. Винограда , был асимптотически самый быстрый из известных алгоритмов умножения матриц с 1990 по 2010 год. В данной работе внимание акцентируется на алгоритме Копперсмита – Винограда и его улучшениях. 

Алгоритм не используется на практике, потому что он дает преимущество только для матриц настолько больших размеров, что они не могут быть обработаны современным вычислительным оборудованием. Если матрица не велика, эти алгоритмы не приводят к большой разнице во времени вычислений. 

Цель лабораторной работы -- теоретическое изучение алгоритма Копперсмита – Винограда, разработка его улучшений и сравнение с классическим алгоритмом на основе полученных в ходе лабораторной работы экспериментальных данных. Для этого необходимо проанализировать изучаемый алгоритм, выделить основные особенности и недостатки для дальнейшей оптимизации. 