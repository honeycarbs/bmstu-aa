\chapter{Исследовательский раздел}

\section{Технические характеристики}

Тестирование выполнялось на устройстве со следующими техническими характеристиками:
\begin{itemize}
	\item Операционная система Ubuntu 20.04.1 LTS;
	\item Память 7 GiB
	\item Процессор Intel(R) Core(TM) i3-8145U CPU @ 2.10GHz \cite{intel}
\end{itemize}
Во время тестирования устройство было подключено к блоку питания и не нагружено никакими приложениями, кроме встроенных приложений окружения, окружением и системой тестирования.

\section{Описание системы тестирования}

Для замеров процессорного времени была написана вставка на языке C, используя пакет cgo \cite{cgo}. Утилита для замера времени приведена на листинге \ref{lst:C}.

\captionsetup{singlelinecheck = false, justification=raggedright}
\begin{lstlisting}[label=lst:C,caption=Вставка на языке С для замера времени]
/*
static double GetCpuTime() {
	struct timespec time;
	if (clock_gettime(CLOCK_PROCESS_CPUTIME_ID, &time)) {
		perror("clock_gettime");
		return 0;
	}
	
	long seconds = time.tv_sec;
	long nanoseconds = time.tv_nsec;
	double elapsed = seconds + nanoseconds*1e-9;
	
	return elapsed;
}
*/
\end{lstlisting}

\section{Таблица времени выполнения алгоритмов}
Результаты профилирования алгоритмов приведены в таблице \ref{tab:time}. Результаты тестирования приведены в таблице \ref{table:testing}. Здесь CW -- алгоритм Копперсмита -- Винограда, ALG -- классический и CW OPT -- оптимизированный Копперсмита -- Винограда.
\begin{table}[H]
	\captionsetup{singlelinecheck = false, justification=raggedleft}
	\caption{Время выполнения алгоритмов}
	\renewcommand{\arraystretch}{1.4}
	\begin{subtable}[h]{0.45\textwidth}
		\centering
		\caption{Четная размерность матриц}
			\begin{tabular}{||l|l|l|l||}
				\hline
				\multirow{2}{*}{n} & \multicolumn{3}{l||}{Время, ns} \\ \cline{2-4} 
				& ALG & CW & CW OPT \\ \hline\hline
				10 & 4.49e-06 & 3.91e-06 & 4.01e-06 \\ \hline 
				20 & 2.74e-05 & 2.24e-05 & 2.30e-05 \\ \hline 
				30 & 8.84e-05 & 6.72e-05 & 7.01e-05 \\ \hline 
				40 & 2.09e-04 & 1.49e-04 & 1.58e-04 \\ \hline 
				50 & 3.96e-04 & 2.98e-04 & 3.14e-04 \\ \hline 
				60 & 6.58e-04 & 4.95e-04 & 5.15e-04 \\ \hline 
				70 & 1.07e-03 & 7.95e-04 & 8.43e-04 \\ \hline 
				80 & 1.60e-03 & 1.20e-03 & 1.22e-03 \\ \hline 
				90 & 2.26e-03 & 1.68e-03 & 1.77e-03 \\ \hline 
				100 & 3.14e-03 & 2.25e-03 & 2.38e-03 \\ \hline 
				150 & 1.06e-02 & 7.88e-03 & 8.33e-03 \\ \hline 
				200 & 2.84e-02 & 2.20e-02 & 2.34e-02 \\ \hline 
				250 & 6.73e-02 & 5.34e-02 & 5.74e-02 \\ \hline 
				300 & 1.14e-01 & 8.97e-02 & 9.60e-02 \\ \hline 
				350 & 1.80e-01 & 1.39e-01 & 1.50e-01 \\ \hline 
				400 & 2.67e-01 & 2.05e-01 & 2.21e-01 \\ \hline 
				450 & 4.17e-01 & 3.23e-01 & 3.48e-01 \\ \hline 
				500 & 5.71e-01 & 4.42e-01 & 4.75e-01 \\ \hline 
			\end{tabular}
		\label{tab:odd}
	\end{subtable}
	\hfill
	\begin{subtable}[h]{0.45\textwidth}
		\centering
		\caption{Нечетная размерность матриц}
			\begin{tabular}{||l|l|l|l||}
				\hline
				\multirow{2}{*}{n} & \multicolumn{3}{l|}{Время, ns} \\ \cline{2-4} 
				& ALG & CW OPT& CW \\ \hline\hline
				11 & 5.17e-06 & 4.56e-06 & 5.05e-06 \\ \hline 
				21 & 3.25e-05 & 2.58e-05 & 2.59e-05 \\ \hline 
				31 & 9.63e-05 & 7.23e-05 & 7.69e-05 \\ \hline 
				41 & 2.14e-04 & 1.64e-04 & 1.73e-04 \\ \hline 
				51 & 4.10e-04 & 3.15e-04 & 3.38e-04 \\ \hline 
				61 & 7.13e-04 & 5.37e-04 & 5.70e-04 \\ \hline 
				71 & 1.11e-03 & 8.36e-04 & 8.79e-04 \\ \hline 
				81 & 1.67e-03 & 1.28e-03 & 1.35e-03 \\ \hline 
				91 & 2.33e-03 & 1.73e-03 & 1.85e-03 \\ \hline 
				101 & 3.23e-03 & 2.36e-03 & 2.49e-03 \\ \hline 
				151 & 1.08e-02 & 8.03e-03 & 8.50e-03 \\ \hline 
				201 & 2.74e-02 & 2.13e-02 & 2.29e-02 \\ \hline 
				251 & 6.80e-02 & 5.27e-02 & 5.66e-02 \\ \hline 
				301 & 1.16e-01 & 9.13e-02 & 9.82e-02 \\ \hline 
				351 & 1.82e-01 & 1.40e-01 & 1.52e-01 \\ \hline 
				401 & 2.58e-01 & 1.99e-01 & 2.14e-01 \\ \hline 
				451 & 4.21e-01 & 3.26e-01 & 3.50e-01 \\ \hline 
				501 & 5.74e-01 & 4.43e-01 & 4.76e-01 \\ \hline 
			\end{tabular}
		\label{tab:even}
	\end{subtable}
	\label{tab:time}
\end{table} 

\section{Графики функций}

\begin{figure}[H]
	\captionsetup{singlelinecheck = false, justification=centering}
	\centering
	\begin{subfigure}[b]{\textwidth}
			\centering
		\begin{tikzpicture}
			\begin{axis}[
				xlabel={размер матрицы},
				ylabel={время, ns},
				width = 0.95\textwidth,
				height=0.3\textheight,
				xmin=0, xmax=600,
				legend pos=north west,
				xmajorgrids=true,
				grid style=dashed,
				]
				\addplot[
				color=blue,
				mark=asterisk,
				]
				table {assets/cw-ev.dat};
				\addplot[
				color=red,
				mark=o,
				]
				table {assets/o-cw-ev.dat};
				\legend{Неоптимизированный, Оптимизированный}
			\end{axis}
		\end{tikzpicture}
	\caption{Четная размерность матрицы}
	\end{subfigure}
	\hfill
	\newline
	\begin{subfigure}[b]{\textwidth}
			\centering
		\begin{tikzpicture}
			\begin{axis}[
				xlabel={размер матрицы},
				ylabel={время, ns},
				width = 0.95\textwidth,
				height=0.3\textheight,
				xmin=0, xmax=600,
				legend pos=north west,
				xmajorgrids=true,
				grid style=dashed,
				];
				
				\addplot[
				color=blue,
				mark=asterisk,
				]
				table {assets/cw-odd.dat};
				\addplot[
				color=red,
				mark=o,
				]
				table {assets/o-cw-odd.dat};
				\legend{Неоптимизированный, Оптимизированный}
			\end{axis}
		\end{tikzpicture}
	\caption{Нечетная размерность матрицы}
	\end{subfigure}
	\hfill
	\caption{Время выполнения алгоритма Копперсмита -- Винограда}
	\label{fig:w-plot}
\end{figure}

Как было доказано в разделе \ref{win-estimate}, оптимизация дает выигрыш по времени практически незначительный. Выигрыш растет с ростом размерности матрицы -- в случае матриц с четной размерностью, выигрыш на матрице $550\times550$ на 7,94\%, а на матрицах с нечетной ($551\times551$) -- на 7,76\%.

\begin{figure}[H]
	\captionsetup{singlelinecheck = false, justification=centering}
	\centering
	\begin{subfigure}[b]{\textwidth}
		\centering
		\begin{tikzpicture}
			\begin{axis}[
				xlabel={размер матрицы},
				ylabel={время, ns},
				width = 0.95\textwidth,
				height=0.3\textheight,
				xmin=0, xmax=600,
				legend pos=north west,
				xmajorgrids=true,
				grid style=dashed,
				];
				\addplot[
				color=blue,
				mark=asterisk,
				]
				table {assets/o-cw-ev.dat};
				\addplot[
				color=red,
				mark=o,
				]
				table {assets/alg-ev.dat};
				\legend{Виноград, Классический}
			\end{axis}
		\end{tikzpicture}
		\caption{Четная размерность матрицы}
	\end{subfigure}
	\hfill
	\begin{subfigure}[b]{\textwidth}
		\centering
		\begin{tikzpicture}
			\begin{axis}[
				xlabel={размер матрицы},
				ylabel={время, ns},
				width = 0.95\textwidth,
				height=0.3\textheight,
				xmin=0, xmax=500,
				legend pos=north west,
				xmajorgrids=true,
				grid style=dashed,
				];
				\addplot[
				color=blue,
				mark=asterisk,
				]
				table {assets/o-cw-odd.dat};
				\addplot[
				color=red,
				mark=o,
				]
				table {assets/alg-odd.dat};
				\legend{Виноград, Классический}
			\end{axis}
		\end{tikzpicture}
		\caption{Нечетная размерность матрицы}
	\end{subfigure}
	\hfill
	\caption{Сравнение времени выполнения алгоритма Копперсмита -- Винограда с классическим матричным}
	\label{fig:w-alg-plot}
\end{figure}
На практике классический показывает результат хуже (\ref{fig:w-alg-plot}), чем алгоритм Копперсмита -- Винограда. Связано это с меньшим количеством более затратных операций. Эта разница, в среднем, составляет 3\%
\begin{figure}[H]
	\captionsetup{singlelinecheck = false, justification=centering}
	\centering
	\begin{subfigure}[b]{\textwidth}
		\centering
		\begin{tikzpicture}
			\begin{axis}[
				xlabel={размер матрицы},
				ylabel={время, ns},
				width = 0.9\textwidth,
				height=0.25\textheight,
				xmin=0, xmax=600,
				legend pos=north west,
				xmajorgrids=true,
				grid style=dashed,
				];
				\addplot[
				color=blue,
				mark=asterisk,
				]
				table {assets/cw-ev.dat};
				\addplot[
				color=red,
				mark=o,
				]
				table {assets/alg-ev.dat};
				\legend{Виноград, Классический}
			\end{axis}
		\end{tikzpicture}
		\caption{Четная размерность матрицы}
	\end{subfigure}
	\hfill
	\begin{subfigure}[b]{\textwidth}
		\centering
		\begin{tikzpicture}
			\begin{axis}[
				xlabel={размер матрицы},
				ylabel={время, ns},
				width = 0.9\textwidth,
				height=0.25\textheight,
				xmin=0, xmax=600,
				legend pos=north west,
				xmajorgrids=true,
				grid style=dashed,
				];
				\addplot[
				color=blue,
				mark=asterisk,
				]
				table {assets/o-cw-odd.dat};
				\addplot[
				color=red,
				mark=o,
				]
				table {assets/alg-odd.dat};
				\legend{Виноград, Классический}
			\end{axis}
		\end{tikzpicture}
		\caption{Нечетная размерность матрицы}
	\end{subfigure}
	\hfill
	\caption{Сравнение времени выполнения оптимизированного алгоритма Копперсмита -- Винограда с классическим матричным}
	\label{fig:o-w-alg-plot}
\end{figure}
Аналогичная ситуации \ref{fig:w-alg-plot} ситуация наблюдается в случае оптимизированного алгоритма (\ref{fig:o-w-alg-plot}) -- алгоритм имеет примерно такой же выигрыш. 
\section{Вывод}
Экспериментально было подтверждено утверждение, установленное в разделе \ref{analyth} -- алгоритм Копперсмита -- Винограда имеет незначительный  выигрыш на матрицах, способных храниться на ЭВМ. Оптимизация стандартного алгоритма так же обеспечивает небольшой выигрыш $\approx 7\%$.	