\addchap{Заключение}
Графовые структуры используются повсеместно -- от алгоритмов нахождения кратчайшего пути до трассировки печатных плат, поэтому возможности его оптимизации исследуются в неисчислимом множестве научных работ. Алгоритм волновой трассировки на параллельных архитектурах показывает существенный прирост только на матрицах, размерность которых больше, чем 100 элементов.
Архитектура устройства, на котором тестировался алгоритм позволяет одному логическому ядру управлять несколькими потоками, поэтому прирост остается ощущаемым на количестве потоков, превышающих количество логических ядер процессора в два раза. Поэтому, ответ на поставленный во введении вопрос неоднозначен -- наличие прироста времени с увеличением потоков зависит не только от количества потоков, но и от архитектуры ЭВМ. 