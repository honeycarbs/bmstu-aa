\addchap{Введение}
\pagenumbering{arabic}
\setcounter{page}{2}

Алгоритм нахождения кратчайшего пути в лабиринте используется во множестве разных областей, начиная от компьютерных игр и заканчивая  трассировкой печатных плат, соединительных проводников и поверхностей микросхем. Хоть алгоритм и прост в реализации, вычислительные расходы на его исполнение достаточно велики. 

В представленной работе исследуется реализация алгоритма Ли на параллельных процессах. Для реализации поставленной задачи следует выполнить следующие этапы:
\begin{itemize}
	\setlength{\itemsep}{1.2pt}
	\setlength{\parskip}{0pt}
	\setlength{\parsep}{0pt}
	\item Проанализировать алгоритм волновой трассировки;
	\item Выделить стратегии распараллеливания алгоритма;
	\item Разработать ПО для решения поставленной задачи;
	\item Сравнить результаты работы последовательного и параллельного алгоритмов с помощью реализованного ПО.
\end{itemize}
Также в данной работе следует ответить на вопрос: "Всегда ли при увеличении количества потоков прирост времени выполнения программы дает выигрыш вдвое?"


Результаты сравнительного анализа будут приведены в виде таблиц и графиков, из чего можно будет сделать вывод об эффективности оптимизаций, предложенных в данной работе и ответить на поставленный вопрос.