\chapter{Технологический раздел}\label{sec:impl}

Раздел содержит листинги реализованных алгоритмов, требование к ПО
и тестирование реализованного программного обеспечения.

\section{Требования к ПО}
Программное обеспечение должно удовлетворять следующим требованиям:
\begin{itemize}
	\item программа получает на вход имя текстового файла со словарем и искомый ключ;
	\item программа выдает значение по искомому ключу. Если ключа в словаре не нашлось, программа выдает текстовую строку \textit{NOT FOUND}.
\end{itemize}

\section{Средства реализации} 
Для реализации ПО был выбран компилируемый многопоточный язык программирования Golang\cite{go}, поскольку язык отличается автоматическим управлением памяти. В качестве среды разработки была выбрана среда VS Code, написание сценариев осуществлялось утилитой make.
\section{Листинги кода}

Листинг \ref{lst:types} демонстрирует реализацию типа <<словарь>> и <<сегментированный словарь>>. Словарь содержит пары значений, описанных структурой <<Entry>>. \\ \\ \\
 
\captionsetup{singlelinecheck = false, justification=raggedright}
\begin{lstlisting}[label=lst:types,caption=Структуры данных]
type Entry struct {
	Key   string
	Value int
}

type Dictionary struct {
	Size    int
	entries []Entry
}

type SgtDictionary struct {
	Size    int
	dicts   []Dictionary
}
\end{lstlisting}
Листинг \ref{lst:brute} демонстрирует реализацию алгоритма полного перебора.
\captionsetup{singlelinecheck = false, justification=raggedright}
\begin{lstlisting}[label=lst:brute,caption=Полный перебор]
func (dct Dictionary) BruteForce(skey string) (int, int, error) {
	compars := 0
	
	for _, entry := range(dct.entries) {
		compars++;
		if (entry.Key == skey) {
			return entry.Value, compars, nil;
		}
	}
	return 0, compars, errors.New("NOT FOUND")
}
\end{lstlisting}

Функция \ref{lst:bin-wrap} является <<оберткой>> для бинарного поиска, вызываемой из пользовательского интерфейса:

\begin{lstlisting}[label=lst:bin-wrap,caption=Бинарный поиск(функция - обертка)]
func (dct Dictionary) BinarySearch(skey string) (int, int, error) {
	dct.sortAlphab()
	
	r, compars, err := dct.__binaryHelper(skey)
	return r, compars, err
}
\end{lstlisting}

Сам бинарный поиск осуществляет функция, демонстрируемая на листинге \ref{lst:bin}. \\

\begin{lstlisting}[label=lst:bin,caption=Бинарный поиск]
func (dct Dictionary) __binaryHelper(skey string) 
					 (result int, compars int, err error) {
	var (
		high    int = len(dct.entries)
		mid     int = high / 2
	)
	switch {
		case high == 0:
		result = 0
		err    = errors.New("NOT FOUND")
		case dct.entries[mid].Key > skey:
		dct.entries = dct.entries[:mid]
		result, compars, err = dct.__binaryHelper(skey)
		case dct.entries[mid].Key < skey:
		dct.entries = dct.entries[mid + 1:]
		result, compars, err = dct.__binaryHelper(skey)
		default:
		result = dct.entries[mid].Value
	}
	compars++
	return
}
\end{lstlisting}

Поиск в сегментированном словаре осуществляется с помощью функции, представленной на листинге \ref{lst:seg-bin}.

\begin{lstlisting}[label=lst:seg-bin,caption=Бинарный поиск в сегментированном словаре]
func (dct Dictionary) SectionedBinary(skey string) (int, int, error) {
	var (
		sec_compars int = 0
		sletter byte = skey[0]
		rvalue  int
		err     error
		compars int
	)
	sdct := dct.SegmentByAlphabet()
	rvalue = 0
	for _, d := range sdct.dicts {
		sec_compars++
		if sletter == d.entries[0].Key[0] {
			rvalue, compars, err = d.BinarySearch(skey)
			break
		}
	}	
	return rvalue, compars + sec_compars, err
}
\end{lstlisting}

\section{Тестирование ПО}
%Результаты тестирования ПО приведены в таблице \ref{table:testing}. 
%\begin{table}[H]
%	\captionsetup{singlelinecheck = false, justification=raggedleft}
%	\caption{Тестирование ПО}
%	\renewcommand{\arraystretch}{2}
%	\begin{center}
%		\begin{tabular}{||c|c|c||}
%			\hline
%			Входные данные & Результат  & Ожидаемый результат \\ \hline\hline
%			peps           & 1          & 1                   \\ \hline
%			warmth         & 1          & 1                   \\ \hline
%			robust         & 1          & 1                   \\ \hline
%			zachetpoAA     & NOT\_FOUND & NOT\_FOUND          \\ \hline
%		\end{tabular}
%	\end{center}
%	\label{table:testing}
%\end{table}

Результаты тестирования ПО приведены в таблице \ref{table:testing}. 
\begin{table}[H]
	\captionsetup{singlelinecheck = false, justification=raggedleft}
	\caption{Тестирование ПО}
	\renewcommand{\arraystretch}{2}
	\begin{center}
		\begin{tabular}{||c|c|c||}
			\hline
			Входные данные & Результат  & Ожидаемый результат \\ \hline\hline
			wanna          & 1          & 1                   \\ \hline
			grab           & 1          & 1                   \\ \hline
			coffee         & -1          & -1                   \\ \hline
			wednesday?    & NOT FOUND & NOT FOUND          \\ \hline
		\end{tabular}
	\end{center}
	\label{table:testing}
\end{table}

\section{Вывод}
Было написано и протестировано программное обеспечение для решения поставленной задачи.