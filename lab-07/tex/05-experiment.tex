\chapter{Исследовательский раздел}\label{sec:exp}
Раздел содержит технические характеристики устройства, на котором проведен эксперимент. Также раздел содержит результаты проведенного эксперимента.
\section{Технические характеристики}
Тестирование выполнялось на устройстве со следующими техническими характеристиками:
\begin{itemize}
	\item операционная система Ubuntu 20.04.1 LTS;
	\item память 7 GiB;
	\item процессор Intel(R) Core(TM) i3-8145U\cite{intel} CPU @ 2.10GHz.
\end{itemize}

\section{Постановка эксперимента}
Эксперимент проведен на данных типа "строка". Количество элементов в словаре фиксировано и равно 2883. Проведенный эксперимент устанавливает зависимость количество сравнений при поиске от позиции элемента в словаре.  \\
Во время тестирования устройство было подключено к блоку питания и не нагружено никакими приложениями, кроме встроенных приложений окружения, окружением и системой тестирования. Оптимизация компилятора была отключена.

\section{Результаты эксперимента}
Результаты эксперимента приведены в приложениях \hyperref[sec:brute]{А}, \hyperref[sec:bin]{Б} и \hyperref[sec:seg]{В} для полного перебора, бинарного поиска и бинарного поиска в сегментированном словаре соответственно.

Медиана подсчитанных количеств сравнений для метода полного перебора равна 1442, для бинарного поиска равна 11 и для бинарного поиска в сегментированном словаре равна 7. 

В среднем, при поиске полным перебором для каждого ключа осуществляется в 131 раз больше сравнений, чем для бинарного поиска и в 206 раз больше сравнений, чем для поиска полным перебором. 

\section{Вывод}\label{sec:exp-sum}
Качественная оценка работы алгоритма зависит от количества сравнений с ключами при поиске. В среднем, при поиске полным перебором для каждого ключа осуществляется в 131 раз больше сравнений, чем для бинарного поиска и в 206 раз больше сравнений, чем для поиска полным перебором. Исходя из результатов эксперимента, можно сделать вывод, что самым оптимальным алгоритмом поиска из трех предложенных является алгоритм бинарного поиска с предварительной сегментацией словаря.