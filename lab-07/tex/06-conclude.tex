\addchap{Заключение}
При повсеместном использовании словарей растет потребность в оптимизации алгоритмов поиска в словаре. В ходе лабораторной работы был проведен анализ алгоритмов поиска полным перебором, бинарного поиска и бинарного поиска с предварительной сегментацией словаря. Результат эксперимента показал, что наиболее качественно задачу решает алгоритм бинарного поиска с предварительной сегментацией словаря. Однако, он требует дополнительных вычислительных затрат на сегментацию словаря и дополнительный объем памяти на хранение выделенных сегментов, что отражает формула \ref{math:mem-seg}. 
Опираясь на проведенное исследование, можно сделать вывод, что самым оптимальным подходом к поиску является разделение словаря на сегменты и осуществление бинарного поиска в каждом сегменте, особенно в случаях, когда словарь, подающийся на вход алгоритму, уже сегментирован. 